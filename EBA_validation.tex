\documentclass[man]{apa7}
\usepackage{authblk}
\usepackage{lipsum}
\usepackage{color,soul}
\usepackage[american]{babel}
\usepackage{csquotes}
\usepackage[style=apa,sortcites=true,sorting=nyt,backend=biber]{biblatex}
\DeclareLanguageMapping{american}{american-apa}
\addbibresource{bibliography.bib}

\title{Development and validation of the Exteroceptive Body Awareness (EBA) questionnaire.}
\shorttitle{EXTEROCEPTIVE BODY AWARENESS QUESTIONNAIRE}


\authorsnames[{1,2},{1,2},{1,2},{2,3}]{Alisha Vabba,Giuseppina Porciello,Maria Serena Panasiti, Salvatore Maria Aglioti} 

\authorsaffiliations{{Department of Psychology, "Sapienza" University of Rome, 00185 Rome, Italy.},{IRCCS, Santa Lucia Foundation, 00142 Rome, Italy.},{Sapienza University of Rome and CLNS@Sapienza, Istituto Italiano di Tecnologia, Viale Regina Elena, 291, 00161, Rome, Italy}}

\leftheader{Weiss}

\abstract{\ The conscious processing of body signals influences higher-order psychological and cognitive processes.  Studies indicate that processing of interoceptive signals (e.g., from internal organs such as heartbeat and breathing) is distinct from that of exteroceptive signals (e.g., from outside the body such as vision, touch, and smell).  Questionnaires are available for assessing interoceptive but not Exteroceptive Body Awareness (EBA). To fill this gap, we developed and validated a   \hl{17-item} scale designed to assess bodily self-consciousness as a consequence of the processing of exteroceptive bodily signals. Exploratory factor analysis was used to select items for inclusion in the scale and assess the psychometric properties and internal reliability of the scale. Furthermore, data are presented for convergent validity through cross-validation with existing body awareness questionnaires and a behavioral measure of visual body awareness. Research applications are discussed within a multi-facet model of exteroceptive and interoceptive awareness as distinct, but at the same time interconnected, dimensions of bodily self-consciousness.}

\authornote{
   \addORCIDlink{Alisha Vabba}{0000-0001-8442-8263}

  Correspondence concerning this article should be addressed to Alisha Vabba, Department of Psychology, "Sapienza" University of Rome, 00185 Rome, Italy.  E-mail: alisha.vabba@uniroma1.it}

\begin{document}

\setlength{\parindent}{10ex}
\maketitle
\noindent The past few decades have shown an increased scientific interest in the role of the body and bodily signals in understanding human cognition and behavior. This led to the development of the Embodied Cognition Theories according to which even higher-other cognitive and emotional processes are grounded in the bodily self \parencite{goldman2009social}. The awareness of owning a body and being the initiators and the controllers of our own actions is known as corporeal awareness \parencite{berlucchi2010body} or Bodily Self-Consciousness (BSC) and relies on the ability of the brain to continuously integrate information about the body originating from different sensory modalities \parencite{blanke2012multisensory}. Although the majority of research on BSC has focused on the role played by exteroceptive body-related signals, namely signals coming from outside the body (such as tactile, visual, and auditory signals), recent research has shown that also interoceptive signals, namely signals originating from inside the body (such as heartbeat, breathing, and gastric function), crucially contribute to BSC \parencite{herbert2012body} and to its stability \parencite{monti2021inside}. These incoming exteroceptive and interoceptive signals are continuously integrated with our prior knowledge and expectations to determine and update our conscious experience of selfhood and even our own identity \parencite{aspell2011multisensory,clark2013whatever,ehrsson201243,park2019coupling,porciello2018enfacement,seth2013interoceptive,seth2016active}. Perhaps the most well-known example of the importance of multisensory integration processes in determining BSC, and consequently its malleability, is the Rubber Hand Illusion (RHI; \Textcite{botvinick1998rubber}) in which the synchronous visual stimulation of a rubber hand and tactile stimulation of the participant’s hidden-from-view hand induces the illusory sensations of owning and controlling the extra-numerary rubber hand. Projecting visual feedback of participants’ heartbeats on a virtual version of a rubber hand can further increase self-identification with the virtual arm, highlighting the role of interoceptive and exteroceptive signals in BSC \parencite{suzuki2013multisensory}.

 A series of behavioral and self-reported measures have been devised to explore individual differences in the capacity to detect both exteroceptive and interoceptive bodily signals, as well as the tendency to focus on - and pay attention to them, an ability called interoceptive accuracy \parencite{garfinkel2015knowing}. Tasks aimed at measuring the capacity to accurately detect specific signals, include: i) cardiac signals, for example via the heartbeat counting task \parencite{schandry1981heart} or heartbeat detection tasks \parencite{azevedo2016participants,whitehead1977relation}; ii) gastric signals, via the measurement of the perception of gastric contractions \parencite{garfinkel2017investigation, herbert2012interoception,van2016water,whitehead1980perception}; iii) respiratory signals via the pneumoception task \parencite{monti2020embreathment} or more classical respiratory resistance tasks 
\parencite{garfinkel2016interoceptive,harver1993signal,steptoe1997perception}; iv) visual signals, via the Body-Scaled Action Task \parencite{guardia2010anticipation,valenzuela2017exteroceptive}; and v) tactile signals via the Somatic Signal Detection Task \parencite{durlik2014being}.
\par Besides behavioral tasks, self-report questionnaires have also been developed to measure different aspects of conscious feelings of corporeal awareness, such as the tendency to notice or pay attention to body sensations and function in normal and stressful conditions, and also aspects related to the emotional and self-regulatory components of reading body signals. This ability has been called interoceptive sensitivity \parencite{garfinkel2015knowing} and is often assessed by means of the Body Perception Questionnaire (BPQ; \Textcite {porges1993body}), the Multidimensional Assessment of Interoceptive Awareness (MAIA-2; \Textcite {mehling2018multidimensional}) and the Body Consciousness Questionnaire (BCQ;\Textcite{miller1981consciousness}).
\par One limitation of existing self-report measures of corporeal awareness is that they either include items related exclusively to interoceptive awareness (e.g., the MAIA) or bulk together across awareness related to different sensory domains (e.g., the BPQ which measures interoceptive and proprioceptive awareness). Therefore, to the best of our knowledge, there are no instruments available which evaluate only the exteroceptive awareness of the body. Indeed, there is evidence that although interoceptive and exteroceptive signals are highly interconnected \parencite{simmons2013keeping}, they cannot be considered as the same construct and they are associated to at least partially different neural counterparts in the brain \parencite{farb2013mindfulness,hurliman2005double}. For instance, performance in the heartbeat counting task does not seem to correlate with tasks measuring body awareness based on exteroceptive cues, such as tactile perception \parencite{durlik2014being} or visual awareness of body size \parencite{valenzuela2017exteroceptive}and interoceptive and exteroceptive attention have opposite effects on subsequent somatic perceptual decision making \parencite{mirams2012interoceptive}.
\par Given the importance of both interoceptive and exteroceptive signals in modulating body representation and higher-order cognitive and psychological processes, it is important to understand not only how these different types of signals are processed in the brain, but also how they are processed in subjective conscious experience. The current paper reports on the development and validation of a new instrument, the Exteroceptive Body Awareness questionnaire (EBA), a self-report scale developed to isolate aspects of bodily self-consciousness related to the processing of exteroceptive body-related signals (e.g., vision, touch, smell). The purpose of the study was to select and retain items for the scale and to assess internal consistency and factor structure. Construct validity was also examined based on convergent and divergent validity with other questionnaires measuring bodily self-awareness, and with behavioral measure of interoceptive and exteroceptive body awareness. 
Scores in the MAIA-2 questionnaire sub-scales, and in Private Body Self-Consciousness, and Private Self-Consciousness, as well as behavioural interoceptive accuracy measured with the heartbeat counting task were used as measures of divergent validity. For convergent validity we included Public Body Self-Consciousness and Public Self-Consciousness, as well as the behavioural measure of exteroceptive body awareness.
\section{Materials and methods}
\subsection{Participants}
\ Participants were recruited via Prolific (www.prolific.co) and from a database of volunteers of the Social and Cognitive Neuroscience Laboratory (SCNLab). They were all fluent Italian speakers. The experimental procedures were approved by the local Ethics Committee of the Department of Psychology, Sapienza University of Rome and were in accordance with the 1964 Declaration of Helsinki. All participants read and signed the informed consent sheet before taking part in the study. All of them were naïve to the purpose of the research and received a monetary compensation for their time.
\subsubsection{Sample 1}
A sample of 371 participants (188 males; \emph{mean age} = 30.73, \emph{standard deviation} = 9.77) volunteers recruited via Prolific participated in the first part of the study, namely the validation of the Exteroceptive Body Awareness (EBA) questionnaire, conceived to measure the individual ability to perceive the exteroceptive dimension of one’s own body. Participants completed the original 33-item EBA online on Survey Monkey (www.surveymonkey.com) alongside a battery of self-report questionnaires measuring different dimensions of bodily self-awareness (see \emph{Measures} for a detailed description).
A total of 55 participants were removed from the statistical analysis, as they did not complete all the questionnaires, failed one or more attention checks, or completed the study in abnormally low time (less than half the average completion time of 15 minutes). Internal reliability and factor structure of the EBA questionnaire were computed on the remaining sample of 316 participants (174 males; \emph{mean age} = 30.66, \emph{standard deviation} = 9.72). 
\subsubsection{Sample 2}
An additional sample of 84 participants (41 males, \emph{mean age} = 24.52, \emph{standard deviation} = 4.98), recruited through the SCNLab voluntary database for another experiment, participated in the second part of this study aimed at characterizing the exteroceptive body awareness construct. Participants completed the reduced 17-item EBA questionnaire as well as other self-report questionnaires measuring different dimensions of bodily self-awareness. To analyze convergent validity between the EBA and the other self-report measures of self-awareness we combined the responses from \emph{Sample 1} (N = 316) and \emph{Sample 2} (N = 84). We eliminated the responses of 15 participants for incomplete survey responses from the whole sample. The final sample for measuring convergent validity consisted of 385 participants (208 males, \emph{mean age} = 28.90, \emph{standard deviation}n = 9.30). Participants from \emph{Sample 2} also underwent objective measures of exteroceptive and interoceptive bodily awareness, namely the Body-Scaled Action task and the heartbeat counting task (see \emph{Measures} for a detailed description)
.The responses of 11 participants were eliminated due to technical issues with the experimental set-up, and the correlation between the EBA total score and the behavioral measure of exteroceptive bodily awareness was calculated on the remaining sample of 71 participants (males = 33, \emph{meand age} = 24.69, \emph{standard deviation} = 4.57).
\subsubsection{Sample 3}
Sofia's sample for confirmatory factor analysis

\subsection{Measures}
\subsubsection{Exteroceptive Body Awareness (EBA) questionnaire development}
An initial pool of 33 items (10 reverse coded) was developed in Italian, with the aim of assessing aspects of bodily self-consciousness related to visual, tactile, and olfactory awareness of the body (see \emph{Supplementary Materials}  for a full list of the 33 original items). Visual items included questions related to the shape and size of the body (e.g. width and height in relation to objects and items of clothing), the coordination of the body in space (e.g. clumsiness and body spatial perception), the visual appearance of the body (e.g. changes in skin color, marks on the skin, and the ability to recognize oneself in mirrors, photos, and videos). The tactile items assessed awareness of tactile sensations on the body by various objects, such as other people, clothing and accessories, insects, food, and sweat. Finally, olfactory items assessed attention towards - and recognition of - the body based on odor (e.g., bad smell). All the items were reviewed for clarity to avoid ambiguity, double negatives, and double-barreled items. The 33 items were presented in random order to participants of \emph{Sample 1}  on Survey Monkey, and factor and reliability analysis led to a final questionnaire comprising 19 items, with good internal reliability ($\alpha$ = \hl{.84}). Items were rated on a 5-point Likert scale ranging from  \emph{1 =“Very uncharacteristic”} to \emph{5 = “Very characteristic”}. Thus, higher scores in the questionnaire reflect a higher awareness of bodily exteroceptive signals. 
\subsubsection{Additional subjective measures of bodily self-awareness} 
The Awareness subscale of the Body Perception Questionnaire, BPQ\parencite{porges1993body} includes 45 items, which reflect the tendency to be aware of different bodily sensations and processes in specific situations, such as swallowing, skin itching, being exhausted, muscle pain, etc. Items are rated on a 5-point Likert scale ranging from \emph{1 = “Never”} to \emph{5 = “Always”}. In our sample the scale showed excellent internal consistency ($\alpha$ = 0.96). 	

The Multidimensional Assessment of Interoceptive Awareness-2 \parencite{mehling2018multidimensional} includes 37 items divided into eight subscales, namely: Noticing ($\alpha$ = .55) includes four items measuring awareness of uncomfortable, comfortable, and neutral body sensations (e.g. “When I am tense I notice where the tension is located in my body”); Not-Distracting ($\alpha$ = .65) includes six items measuring the tendency not to ignore or distract oneself from sensations of pain or discomfort (e.g. “I distract myself from sensations of discomfort”); Not-Worrying ($\alpha$ = .71) includes five items measuring the tendency not to worry or experience emotional distress with sensations of pain or discomfort (e.g. “I start to worry that something is wrong if I feel any discomfort”); Attention Regulation ($\alpha$ = .79) includes seven items measuring the ability to sustain and control attention to body sensations (e.g. “I can pay attention to my breath without being distracted by things happening around me”); Emotional Awareness ($\alpha$ = .77) includes five items measuring awareness of the connection between body sensations and emotional items (e.g. “I notice how my body changes when I am angry”); Self-Regulation ($\alpha$ = .78) includes four items measuring the ability to regulate distress by attention to body sensations (e.g. “When I feel overwhelmed I can find a calm place inside”); Body Listening ($\alpha$ = .69) includes three items measuring active listening to the body for insight (e.g. “I listen for information from my body about my emotional state”); Trusting ($\alpha$ = .79) includes three items measuring experience of one’s body as safe and trustworthy  (e.g. “I am at home in my body”). Items were rated on a 6-point Likert scale ranging from \emph{0 = “Never”} to \emph{5 = “Always”}. 

The Body Consciousness Questionnaire \parencite{miller1981consciousness} includes 15 items divided into three sub-scales. Private Body Consciousness ($\alpha$ = .73) includes five items measuring attention towards internal body states (e.g. “I am sensitive to internal bodily tensions”); Public Body Consciousness ($\alpha$ = .64) includes seven items measuring attention to the social presentation of the body (e.g. “When with others, I want my hands to be clean and look nice”); and Body Competence ($\alpha$ = .56) includes four items measuring the self-evaluation of bodily competence (e.g. “For my size, I'm pretty strong.”). Items are rated on a 5-point Likert scale ranging from \emph{0 = “extremely uncharacteristic”} to \emph{4 = “extremely characteristic”}. 

The Self-Consciousness Scale \parencite{scheier1985self} includes 22 items divided into three subscales. Private Self-Consciousness ($\alpha$ = .72) includes nine items measuring the tendency to be introspective and to attend to inner thoughts and feelings (e.g. “I’m always trying to figure myself out”); Public Self-Consciousness ($\alpha$ = .78) includes seven items measuring attention to the self as visible to others (e.g. “I care a lot about how I present myself to others”); and Social Anxiety ($\alpha$ = .82) includes six items measuring fear and anxiety in social situations (e.g. “It takes me time to get over my shyness in new situations”). Items are rated on a 4-point Likert scale ranging from \emph{0 = “not like me at all”} to \emph{3 = “a lot like me”}.  
\subsubsection{Objective measure of exteroceptive body self-awareness:The Body-Scaled Action task} 
As an objective measure of exteroceptive body awareness, participants \emph{Sample 2} completed a modified version of the Body-Scaled Action task (Guardia et al., 2010). At the beginning of the experimental session, the experimenter measured each participant’s height and shoulder width. These parameters were entered in the \emph{E-prime 2} (Psychology Software Tools, Pittsburgh, PA) script to create a series of personalized visual stimuli (i.e., open doors) which were projected on a white wall during the experiment. Participants stood at a 5-meter distance from the wall and their task consisted in judging whether their body could pass through the series of projected doors. The task was composed by two experimental blocks, namely Body Width and Body Height, each composed by 21 trials and 4 practice trials. In the “Body Width” block participants observed a series of doors that varied in width based on the participant’s actual shoulder width. Specifically, doors varied in steps of 1cm up to 10 cm larger or thinner than participant’s actual shoulder width. In this block, the door height was fixed at 20 cm taller than the participants’ actual height. Participants observed each door with no time constraints and answered whether they could pass through the door without turning sideways, by selecting the answer “yes” of “no” from a keyboard placed in front of them. The “Body Height” block was similar to the “Body Width” block but in this case the doors varied in height based on participant’s actual height, i.e., they varied in steps of 1 cm up to 10 cm taller or shorter than the participant’s actual height. In this block, the door width was fixed at 20 cm larger than the participant’s actual shoulder width and they judged whether they could pass through the door without bending. The order of trials and blocks was randomized for all participants. At the end of the task, participants were asked to judge their perceived accuracy in each block, using a visual-analogue scale (VAS) ranging from \emph{0 = “not at all accurate”} to \emph{100 = "completely accurate accurate”}. The average of the two responses was considered a measure of meta-cognitive exteroceptive bodily awareness. 
\subsubsection{Objective measure of interoceptive body awareness: the heartbeat counting task (HCT)} 
As a measure of interoceptive accuracy, participants completed the heartbeat counting task (HCT; Schandry, 1981). In four randomized intervals lasting 25, 35, 45, and 100 seconds, participants were asked to mentally count their real heartbeats without relying on external cues (e.g., taking their pulse). The time intervals were delimited by two auditory tones delivered through a set of headphones. Participants were explicitly asked not to estimate the number of heartbeats they believed occurred between the two tones, but to only count heartbeats they truly perceived. At the end of each trial, participants reported the number of felt heartbeats using the keyboard. During the task, participants’ real heartbeats were recorded using a two-electrode portable custom-made logger for ECG signals (MyHeart). At the end of the four blocks, participants were also asked to judge their perceived confidence in the task performance, using a visuo-analogue scale (VAS) ranging from 0 = “not confident at all” to 100 = “extremely confident”. Instructions, trial order, and response collection were handled by E-Prime 2 software (Psychology Software Tools, Pittsburgh, PA). A schematic figure of the heartbeat counting task is shown in Figure 1.
\subsection{Data handling}
\subsubsection{EBA questionnaire validation}
Internal reliability and factor structure of the EBA questionnaire were computed on the responses of participants from Sample 1 to the initial pool of 33 items. Initially, an r-matrix containing polychoric correlations between all pairs of items was performed to check the pattern of item associations and to identify any items that were not meaningfully related to the others and that should be excluded. After checking that the remaining items’ internal reliability was acceptable (alpha values higher than 0.70 according to \Textcite {bland1997difference} and assumptions were respected (\emph{KMO} measure of sampling adequacy and Bartlett’s test of sphericity), parallel analysis \parencite{horn1965rationale} was used to generate datasets based on permutations of the data and to suggest the number of factors for extraction. To select items for inclusion in the final scale, and to examine the psychometric properties of the questionnaire, exploratory factor analysis with principal axis factoring extraction and promax oblique rotation (as we expected items to share variance) was performed on the data, and factor loadings below 0.4 (Stevens, 2009) were suppressed. The procedure was repeated twice until no items needed to be excluded for low factor loadings. Internal consistency for the final scale and the identified factors was assessed using Cronbach’s $\alpha$ \parencite{bland1997difference}. To examine convergent and divergent validity, using the combined responses from participants in \emph{Sample 1} and \emph{Sample 2}, the EBA scores for the final questionnaire (see \emph{Table 1}) was correlated with the additional self-report measures of body awareness (i.e., BPQ, BCQ, SCS, and MAIA-2 questionnaires).
\subsubsection{Characterization of the exteroceptive body awareness construct }
To test the relationship between behavioral, self-reported, and meta-cognitive measures of exteroceptive body awareness, in the responses of participants from \emph{Sample 2} we measured whether the total score of the 17-item EBA correlated with the objective measure of exteroceptive awareness i.e., the modified version of the Body-Scaled Action task \parencite{guardia2010anticipation}. To compute an individual score for the Body-Scaled Action task, we used signal detection theory \parencite{stanislaw1999calculation}. Participants’ responses were categorized as \emph{hits} (participants accurately judge that they can pass through the door), \emph{misses} (participants judge that they cannot pass through the door when they can), \emph{false alarms} (participants judge that they can pass through the door when they actually cannot), and \emph{correct rejections} (participants accurately judge that they cannot pass through the door). \emph{Hit rate} and \emph{false alarm rate} were calculated and used to compute the d’ as follows: 

 \[ d^{\prime} = z(Hits) - z(False \, alarms)\]
 The individual \emph{d’} was considered an individual measure of exteroceptive bodily accuracy. Finally, individual scores reported at the 0-100 VAS on perceived confidence given at the end of the Body-Scaled Action task were considered meta-awareness measures of exteroceptive body awareness. All statistical analyses for the experiment were run via SPSS (IBM) and R (R Development Core Team 2013, packages \emph{psych, ltm, Hmisc., psycho, dplyr, EFA.dimensions)}.
 \subsubsection{Interoceptive accuracy}
A Matlab (The MathWorks, Inc) custom script was used to identify and count the number of R-wave peaks on the ECG trace, which was also visually inspected for artefacts. Interoceptive accuracy was then calculated as the ratio of perceived to real heartbeats averaged across all trials, using the following formula: 
Interoceptive accuracy = ¼ ∑ (1 – [|recorded heartbeats—counted heartbeats|]/recorded heartbeats). 
Thus, scores closer to 1 indicated higher performance in the HCT. 
Interoceptive meta-awareness was calculated as the difference between confidence judgements (scores in the final VAS) measured on the overall performance at the HCT and the interoceptive accuracy score, using the following formula:
Interoceptive meta-awareness = [|confidence ratings — interoceptive accuracy|]/ interoceptive accuracy). 
Therefore, scores closer to 0 indicated greater meta-awareness of the capacity to correctly count heartbeats (i.e., low discrepancy between the accuracy score and the beliefs about one’s own interoceptive abilities). 
	Interoceptive sensibility was the mean score in the Notice subscale of the MAIA.


\section{Results}
\subsection{Factor structure and internal consistency of the EBA questionnaire}
Based on inspection of the r-matrix containing Pearson correlations between all 33 items, we eliminated 15 items that showed non-significant correlations with the others. For the remaining set of 18 items, internal reliability was high ($\alpha$ = .85) and correlations between each item and the total score calculated as the sum of all items were well above the suggested cutoff of 0.3 \parencite{streiner2015health}. The determinant of coefficient was 0.0034388 which is > 0.00001, the \emph{KMO} measure of sampling adequacy of 0.89 was meritorious \parencite{kaiser1974little} and Bartlett’s test of sphericity was significant ($\chi$ = 1748.115, \emph{p} < .001) indicating suitability of the data for exploratory factor analysis. Parallel analysis based on 2000 permutations of the real data, suggested the extraction of two factors. The sample size was good for factor analysis \parencite{comrey2013first}. Exploratory factor analysis with principal component extraction was performed on the dataset and promax oblique rotation was used as we assumed the factors to be correlated. One item was suppressed as it showed cross-loadings across factors. The remaining 17 items showed an internally consistent structure (\emph{mean} = 61.48, \emph{standard deviation} = 8.42, $\alpha$ = .84) and contained items related to the capacity of accurately detecting signals from the body related to visual, tactile, and olfactory cues. Two factors explained 36.61\% of the cumulative variance. The first factor (\emph{mean} = 34.87, \emph{standard deviation} = 4.74, $\alpha$ = .78) was composed of 9 items and was called “Exteroceptive Body Awareness” (e.g., “I can immediately tell if I will be able to reach an object on a high shelf, without using a support”).  The second factor (\emph{mean} = 26.61, \emph{standard deviation} = 4.84, $\alpha$ = .74) contained 8 items and was called “Social Exteroceptive Body Awareness (e.g., “I can immediately tell when I start to have a bad body odour”). Factor loadings for all items constituting the 17-item questionnaire are listed in \emph{Table 1}. 
\subsection{Convergent validity of the EBA questionnaire}
To analyze convergent validity, the EBA total score was correlated with the additional self-report measures of body and self-awareness (i.e., BPQ, BCQ, SCS, and MAIA-2 questionnaires). The obtained Pearson’s correlations of the EBA with the other self-report measures are reported in \emph{Table 2}, as well as means, standard deviations, and Cronbach’s alpha values for each measured sub-scale. The EBA showed a significant correlation with the Awareness subscale of the BPQ (\emph{r} = .22, \emph{p} < 0.001), with  several sub-scales of the MAIA – Noticing (\emph{r} = .34, \emph{p} < .001), Attention Regulation (\emph{r} = .29, \emph{p} < .001), Emotional Awareness (\emph{r} = .26, \emph{p} < .001), Self-Regulation (\emph{r} = .22, \emph{p} < .001), Body Listening (\emph{r} = .24, \emph{p} < .001), and Trusting (\emph{r} = .26, \emph{p} < .001); with all sub-scales of the Body Self-Consciousness Questionnaire - Private Body Consciousness (\emph{r} = .36, \emph{p} < .001), Public Body Consciousness (\emph{r} = .21, \emph{p} < .001)¸and Body Competence (\emph{r} = .38, \emph{p} < .001), and with the Private Self-Consciousness (\emph{r} = .27, \emph{p} < .001) and Public Self-Consciousness sub-scales of the Self-Consciousness Scale  (\emph{r} = .21, \emph{p} < .001). However, the EBA did not correlate with the Not-Distracting (\emph{r}= -.07,02, \emph{p} = .168) and the Not-Worrying (\emph{r}= .02, \emph{p} = .666) subscales of the MAIA, or with the Social Anxiety (\emph{r}= -.06, \emph{p} = .256) sub-scale of the Self Consciousness Scale. All reported significant correlations were below the Bonferroni cutoff for multiple comparisons (\emph{p} = .003).
\subsection{Characterization of the exteroceptive body awareness construct }
The individual\emph{d’} was considered an individual measure of exteroceptive bodily accuracy (\emph{mean}= 0.3, \emph{standard deviation} = 1.05). Finally, individual scores reported at the 0-100 VAS on perceived confidence given at end of the Body-Scaled Action task were considered meta-awareness measures of exteroceptive body awareness (\emph{mean} = .89, \emph{standard deviation} = 0.23). Correlation analysis showed that the 17-item EBA scale (\emph{mean} = 60.51, \emph{standard deviation} = 6.86) did not significantly correlate with the objective measure of exteroceptive body accuracy (\emph{r} = 0.11; \emph{p} = .36) or with exteroceptive meta-awareness (\emph{r} = 0.16; \emph{p} = .18). The objective measure of exteroceptive awareness and the measure of meta-awareness also did not correlate (\emph{r} = 0.00; \emph{p} = .999). This result suggests that, like interoception, exteroceptive bodily awareness is a multidimensional construct. 

\section{Discussion}


Table~\ref{tab:BasicTable} summarizes the data. 

\begin{table}
  \caption{Sample Basic Table}
  \label{tab:BasicTable}
  \begin{tabular}{@{}llr@{}}         \toprule
  \multicolumn{2}{c}{Item}        \\ \cmidrule(r){1-2}
  Animal    & Description & Price \\ \midrule
  Gnat      & per gram    & 13.65 \\
            & each        &  0.01 \\
  Gnu       & stuffed     & 92.50 \\
  Emu       & stuffed     & 33.33 \\
  Armadillo & frozen      &  8.99 \\ \bottomrule
  \end{tabular}
\end{table}

\begin{figure}
    \caption{This is my first figure caption.}
    \includegraphics[bb=0in 0in 2.5in 2.5in, height=2.5in, width=2.5in]{Figure1.pdf}
    \label{fig:Figure1}
\end{figure}

Figure~\ref{fig:Figure1} shows this trend. \lipsum[16]


\section{Instrument}
\label{app:instrument}

As shown in Figure~\ref{fig:Figure2}, these results are impressive. 

\begin{figure}
    \caption{This is my second figure caption.}
    \includegraphics[bb=0in 0in 2.5in 2.5in, height=2.5in, width=2.5in]{Figure1.pdf}
    \label{fig:Figure2}
\end{figure}


\section{Pilot Data}
\label{app:surveydata}

The detailed results are shown in Table~\ref{tab:DeckedTable}.
\begin{table}
  \begin{threeparttable}
    \caption{A More Complex Decked Table}
    \label{tab:DeckedTable}
    \begin{tabular}{@{}lrrr@{}}         \toprule
    Distribution type  & \multicolumn{2}{l}{Percentage of} & Total number   \\
                       & \multicolumn{2}{l}{targets with}  & of trials per  \\
                       & \multicolumn{2}{l}{segment in}    & participant    \\ \cmidrule(r){2-3}
                                    &  Onset  &  Coda            &          \\ \midrule
    Categorical -- onset\tabfnm{a}  &    100  &     0            &  196     \\
    Probabilistic                   &     80  &    20\tabfnm{*}  &  200     \\
    Categorical -- coda\tabfnm{b}   &      0  &   100\tabfnm{*}  &  196     \\ \midrule
    \end{tabular}
    \begin{tablenotes}[para,flushleft]
        {\small
            \textit{Note.} All data are approximate.

            \tabfnt{a}Categorical may be onset.
            \tabfnt{b}Categorical may also be coda.

            \tabfnt{*}\textit{p} < .05.
            \tabfnt{**}\textit{p} < .01.
         }
    \end{tablenotes}
  \end{threeparttable}
\end{table}

\end{document}

%% 
%% Copyright (C) 2019 by Daniel A. Weiss <daniel.weiss.led at gmail.com>
%% 
%% This work may be distributed and/or modified under the
%% conditions of the LaTeX Project Public License (LPPL), either
%% version 1.3c of this license or (at your option) any later
%% version.  The latest version of this license is in the file:
%% 
%% http://www.latex-project.org/lppl.txt
%% 
%% Users may freely modify these files without permission, as long as the
%% copyright line and this statement are maintained intact.
%% 
%% This work is not endorsed by, affiliated with, or probably even known
%% by, the American Psychological Association.
%% 
%% This work is "maintained" (as per LPPL maintenance status) by
%% Daniel A. Weiss.
%% 
%% This work consists of the file  apa7.dtx
%% and the derived files           apa7.ins,
%%                                 apa7.cls,
%%                                 apa7.pdf,
%%                                 README,
%%                                 APA7american.txt,
%%                                 APA7british.txt,
%%                                 APA7dutch.txt,
%%                                 APA7english.txt,
%%                                 APA7german.txt,
%%                                 APA7ngerman.txt,
%%                                 APA7greek.txt,
%%                                 APA7czech.txt,
%%                                 APA7turkish.txt,
%%                                 APA7endfloat.cfg,
%%                                 Figure1.pdf,
%%                                 shortsample.tex,
%%                                 longsample.tex, and
%%                                 bibliography.bib.
%% 
%%
%% End of file `./samples/longsample.tex'.